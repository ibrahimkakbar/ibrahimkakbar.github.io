\documentclass[a4paper]{article}

% Language and Font Encoding
\usepackage[english]{babel}
\usepackage[utf8x]{inputenc}
\usepackage[T1]{fontenc}
\usepackage{listings}
\usepackage{float}
\usepackage{caption}         % For captions
\usepackage{fancyhdr}        % For header and footer
\usepackage{rotating}
\usepackage{amssymb}
\usepackage{varwidth}


% Page size and margins
\usepackage[a4paper,top=3cm,bottom=2cm,left=2cm,right=2cm,marginparwidth=1.75cm]{geometry}

% Useful Packages
\usepackage{amsmath}
\usepackage{graphicx}
\usepackage{tikz}
\usetikzlibrary{automata, positioning}
\usepackage{amsthm}

\DeclareMathOperator*{\argmax}{argmax}
\DeclareMathOperator*{\argmin}{argmin}
\DeclareMathOperator*{\softmax}{softmax}

\newcommand{\squeezeup}{\vspace{-2.5mm}}
\newcommand{\norm}[1]{\left\lVert#1\right\rVert}

\usepackage[colorinlistoftodos]{todonotes}
\usepackage[colorlinks=true, allcolors=blue]{hyperref}

\pagestyle{fancy}
\lhead{Linear Algebra Notes}
\rhead{}

\begin{document}

\title{Linear Algebra: Vector Spaces, Subspaces, and Linear Maps}
\author{Ibrahim Akbar\\}
\maketitle

%--------------------------------------------------------------------------------------------------------Vector Space--------------------------------------------------------------------------------------------------------------------
\section{Vector Space}
\textbf{Definition:} A collection, $\mathcal{V}$, is known as a vector (linear) space over the field $\mathbb{F}$ if it is equipped with the operations $+:\mathcal{V}\times\mathcal{V} \rightarrow \mathcal{V}$ and $\cdot:\mathbb{F}\times\mathcal{V}\rightarrow\mathcal{V}$ such that the following properties hold.\\

\begin{itemize}
\item Closure under Vector Addition:  $\mathbf{u + v} \in \mathcal{V} \quad \forall \mathbf{u,v} \in \mathcal{V}$
\item Closure under Scalar Multiplication:  $\alpha \mathbf{v} \in \mathcal{V}  \quad \forall \mathbf{v} \in \mathcal{V}, \alpha \in \mathbb{F}$
\item $(\alpha + \beta)\mathbf{v} = \alpha \mathbf{v} + \beta \mathbf{v}$
\item $\alpha(\beta\mathbf{v}) = (\alpha\beta)\mathbf{v}$
\item $\alpha(\mathbf{u + v}) = \alpha\mathbf{u} + \alpha\mathbf{v} $
\item $\exists \{\mathbf{0}\} \hspace{2mm} \textnormal{s.t.} \hspace{2mm} \mathbf{v} +\{\mathbf{0}\} = \mathbf{v} \quad \forall\mathbf{v}\in\mathcal{V}$
\item For each $\mathbf{v}\in\mathcal{V}\quad\exists \mathbf{(-v)}\hspace{2mm} \textnormal{s.t.} \hspace{2mm} \mathbf{v} +\mathbf{(-v)} = \{\mathbf{0}\}$
\item $1\in\mathbb{F},\mathbf{v}\in\mathcal{V} \Rightarrow \mathbf{1v} = \mathbf{v}$
\end{itemize}

\subsection{Examples}
Some examples of vector spaces are: $\mathbb{R}^{n}, \mathbb{C}^{n}$

%----------------------------------------------------------------------------------------------------------Sub Space --------------------------------------------------------------------------------------------------------------------
\section{Subspace}
\textbf{Definition:} A subspace, $\mathcal{U}$ of a vector space, $\mathcal{V}$, over the field, $\mathbb{F}$, is a non-empty set such that,

\begin{itemize}
\item $\mathbf{u,v}\in\mathcal{W}\Rightarrow\mathbf{u+v}\in\mathcal{W}$
\item $\mathbf{v}\in\mathcal{W}\Rightarrow\alpha\mathbf{w}\in\mathcal{W},\forall\alpha\in\mathbb{F}$
\item $\exists \{\mathbf{0}\} \hspace{2mm} \textnormal{s.t.} \hspace{2mm} \mathbf{w} +\{\mathbf{0}\} = \mathbf{w} \quad \forall\mathbf{w}\in\mathcal{W}$
\end{itemize}

\subsection{Examples}
Some examples of subspaces are:

$$
V := \{x\in\mathbb{R}^{n}\mid x^{2} \geq 4x\}
$$

Note that $V' := \{x\in\mathbb{R}^{n}\mid x^{2} > 4x\}$ would not be a subspace nor would any polynomial that does not equal zero. Why?\\
In general, all spaces in $\mathbb{R}^{n}$ that are defined by a homogeneous system of linear equations adhere to these properties. For a visual/geometric representation it is a point, line, plane, hyper-plane that goes through the origin $\{\mathbf{0}\}$.

%----------------------------------------------------------------------------------------------------Linear Independence-----------------------------------------------------------------------------------------------------------------
\section{Linearly Independent}
\textbf{Definition:} A set of vectors $\{\mathbf{x}_{1},\mathbf{x}_{2},\ldots,\mathbf{x}_{n}\}$ is said to be linearly independent if:
$$
\sum_{i = 1}^{n}\alpha_{i}\mathbf{x}_{i} = 0 \iff \alpha_{i} = 0 \hspace{2mm} \forall i
$$

The vectors cannot be written as a linear combination of each other and thus cannot describe each other. Therefore it is the same as saying there is no redundant information in the vectors.

\subsection{Example}
Let $\mathcal{S} = \Bigg\{\begin{bmatrix}2\\1\\-1\\\end{bmatrix},\begin{bmatrix}3\\2\\-1\\\end{bmatrix},\begin{bmatrix}-4\\-2\\1\\\end{bmatrix}\Bigg\}$, then we see that $-2v_{1} + 0v_{2} = v_{3}$ and therefore $\mathcal{S}$ is not a linearly independent set.\\

%------------------------------------------------------------------------------------------------------------Span------------------------------------------------------------------------------------------------------------------------
\section{Span}
\textbf{Definition:} The span of a set of vectors $\mathcal{S} = \{\mathbf{x}_{1}, \mathbf{x}_{2}, \ldots, \mathbf{x}_{n}\}$ is the set of vectors, $\mathcal{V}$, that is described by all possible linear combinations of $\mathcal{S}$.
$$
V := span(\mathcal{S}) = \{y \mid y=\sum_{i = 1}^{n}\alpha_{i}x_{i}\hspace{2mm}\forall\alpha_{i}\in\mathbb{F}\}
$$ 

\subsection{Examples}
Given the vector $v = \begin{bmatrix}1\\0\\\end{bmatrix}$ over the field of all reals it would span the space 
$$
\mathcal{S} = \bigg\{\begin{bmatrix}\alpha\\0\\\end{bmatrix}\mid \alpha\in\mathbb{R}\bigg\}
$$
Lets also consider $u = \begin{bmatrix}0\\1\\\end{bmatrix}$ in the same field. Together these would span
$$
\mathcal{T} = \bigg\{\begin{bmatrix}\alpha\\\beta\\\end{bmatrix}\mid\alpha,\beta\in\mathbb{R}\bigg\} = \mathbb{R}^{2}
$$

%------------------------------------------------------------------------------------------------------------Basis-----------------------------------------------------------------------------------------------------------------------
\section{Basis}
\textbf{Definition:} A basis for a vector space, $\mathcal{V}$, is a set of vectors, $\mathcal{T}$, such that it is the maximal linear independent set and minimal spanning set of $\mathcal{V}$.

% Existence of a Basis and Uniqueness of a Basis
Is a basis for a space unique?\\
Let their be two bas
Does a basis for a space exist?\\
\subsection{Examples}
A basis for $\mathbb{R}^{2}$ is $\begin{bmatrix} 0 & 1\\ 1 & 0\\ \end{bmatrix}$.\\

We need to show that the set of vectors are linearly independent and span the space of $\mathbb{R}^{2}$.
$$
\alpha_{1}v_{1} + \alpha_{2}v_{2} = 0 \iff \alpha_{i} = 0\hspace{2mm}i = \{1,2\}
$$
$$
\alpha_{1}\begin{bmatrix}0\\1\\\end{bmatrix} + \alpha_{2}\begin{bmatrix}1\\0\\\end{bmatrix} = 0
$$
$$
\Rightarrow
\begin{bmatrix}0\\\alpha_{1}\end{bmatrix} + \begin{bmatrix}\alpha_{2}\\0\\\end{bmatrix} = 0 
$$
$$
\Rightarrow
\begin{bmatrix}\alpha_{2}\\\alpha_{1}\\\end{bmatrix} = 0 \Rightarrow \alpha_{1} = \alpha_{2} = 0
$$

To see that it spans $\mathbb{R}^{2}$ can be observed from the previous section.\\
%----------------------------------------------------------------------------------------------------------Linear Maps-------------------------------------------------------------------------------------------------------------------
\section{Linear Maps}
\textbf{Definition:} Given vector spaces, $\mathcal{U,V}$ over the field $\mathbb{F}$ a linear map, $T: \mathcal{U}\rightarrow\mathcal{V}$ is a transformation that preserves additivity and homogeneity.

\begin{itemize}
\item $T(\mathbf{u + w}) = T(\mathbf{u}) + T(\mathbf{w})\hspace{2mm} \forall\mathbf{u,w}\in\mathcal{U}$
\item $T(\alpha\mathbf{u}) = \alpha T(\mathbf{u})$
\end{itemize}

\subsection{Examples}
\begin{itemize}
\item $T(x): x\rightarrow x^{2}$ is not linear.
\item $T(x): x\rightarrow 0$ is linear.
\end{itemize}

If $\mathcal{U,V}$ are finite dimensional and a basis exists for them, the linear maps can be defined as matrices.

\subsection{Range}
\textbf{Definition:} Let $T: \mathcal{V}\rightarrow\mathcal{U}$ be a linear map and denote the Range(Image) of $T$ as,
$$
\mathcal{R}(T) = \{\mathbf{u}\mid T(v) = u,\hspace{2mm} \mathbf{u}\in\mathcal{U},\hspace{1mm}\mathbf{v}\in\mathcal{V}\}
$$

%-Is the range space unique? Does the range space exist?

\subsection{Nullspace}
\textbf{Definition:} Let $T: \mathcal{V}\rightarrow\mathcal{U}$ be a linear map, then denote the Nullspace of $T$ as,
$$
\mathcal{N}(T) = \{\mathbf{v}\mid T(v) = 0,\hspace{2mm}\mathbf{u}\in\mathcal{U}\}
$$

% Is the nullspace unique?

\end{document}







